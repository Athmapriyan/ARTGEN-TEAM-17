\documentclass{article}
\usepackage{hyperref}

\title{Setup Instructions for Competitive Programming Platform}
\author{}
\date{}

\begin{document}

\maketitle

\section{Prerequisites}
Ensure you have the following installed:
\begin{itemize}
    \item Python 3.8+
    \item Docker
    \item PostgreSQL or SQLite
\end{itemize}

\section{Installation Steps}

\subsection{Step 1: Clone the Repository}
\begin{verbatim}
git clone https://github.com/yourusername/competitive-programming-platform.git
cd competitive-programming-platform
\end{verbatim}

\subsection{Step 2: Create a Virtual Environment}
\begin{verbatim}
python -m venv venv
source venv/bin/activate  # For Linux/macOS
venv\Scripts\activate      # For Windows
\end{verbatim}

\subsection{Step 3: Install Dependencies}
\begin{verbatim}
pip install -r requirements.txt
\end{verbatim}

\subsection{Step 4: Apply Migrations}
\begin{verbatim}
python manage.py makemigrations
python manage.py migrate
\end{verbatim}

\subsection{Step 5: Create Superuser}
\begin{verbatim}
python manage.py createsuperuser
\end{verbatim}
Follow the prompts to set up an admin account.

\subsection{Step 6: Run the Development Server}
\begin{verbatim}
python manage.py runserver
\end{verbatim}
The application will be accessible at \url{http://127.0.0.1:8000/}.

\section{Usage}
\begin{itemize}
    \item \textbf{Admin Panel:} Access at \texttt{/admin/} to manage users, problems, and contests.
    \item \textbf{Submit Code:} Use the API at \texttt{/submit/} to submit solutions.
    \item \textbf{Leaderboard:} View rankings at \texttt{/leaderboard/}.
\end{itemize}

\section{Deployment}
To deploy the application, consider using:
\begin{itemize}
    \item Gunicorn \& Nginx for production setup
    \item AWS/DigitalOcean for hosting
    \item Docker Compose for containerized deployment
\end{itemize}

\section{License}
This project is licensed under the MIT License.

\section{Contributors}
\begin{itemize}
    \item Your Name
    \item Other Contributors
\end{itemize}

For contributions, open an issue or submit a pull request.

\end{document}
